\documentclass{beamer}

\usepackage{beamerthemesplit}
\usepackage[utf8x]{inputenc}
\usepackage{pgf}
\usepackage{default}
\usepackage{url}
\usepackage{subfigure}
\usepackage{algorithmic} 
\usepackage{algorithm} 
\usepackage{amssymb}
\usepackage{graphicx}
\usepackage{booktabs}
\usepackage{mathtools}

\usetheme{Singapore}


%Define some commands for printing correct variables in math mode 
\newcommand{\av}{\textbf{a}}
\newcommand{\bv}{\textbf{b}}
\newcommand{\cv}{\textbf{c}}
\newcommand{\dv}{\textbf{d}}
\newcommand{\ev}{\textbf{e}}
\newcommand{\fv}{\textbf{f}}
\newcommand{\gv}{\textbf{g}}
\newcommand{\hv}{\textbf{h}}
\newcommand{\iv}{\textbf{i}}
\newcommand{\jv}{\textbf{j}}
\newcommand{\kv}{\textbf{k}}
\newcommand{\lv}{\textbf{l}}
\newcommand{\mv}{\textbf{m}}
\newcommand{\nv}{\textbf{n}}
\newcommand{\ov}{\textbf{o}}
\newcommand{\pv}{\textbf{p}}
\newcommand{\qv}{\textbf{q}}
\newcommand{\rv}{\textbf{r}}
\newcommand{\sv}{\textbf{s}}
\newcommand{\tv}{\textbf{t}}
\newcommand{\uv}{\textbf{u}}
\newcommand{\vv}{\textbf{v}}
\newcommand{\wv}{\textbf{w}}
\newcommand{\xv}{\textbf{x}}
\newcommand{\yv}{\textbf{y}}
\newcommand{\zv}{\textbf{z}}

\newcommand{\alphav}{\mbox{\boldmath$\alpha$}}
\newcommand{\betav}{\mbox{\boldmath$\beta$}}
\newcommand{\gammav}{\mbox{\boldmath$\gamma$}}
\newcommand{\xiv }{\mbox{\boldmath$\xi$}}
\newcommand{\muv}{\mbox{\boldmath$\mu$}}
\newcommand{\tauv}{\mbox{\boldmath$\tau$}}
\newcommand{\Omegam}{\mbox{\boldmath$\Omega$}}
\newcommand{\Lambdam}{\mbox{\boldmath$\Lambda$}}
\newcommand{\Sigmam}{\mbox{\boldmath$\Sigma$}}
\newcommand{\Gammam}{\mbox{\boldmath$\Gamma$}}
\newcommand{\Deltam}{\mbox{\boldmath$\Delta$}}
\newcommand{\Thetam}{\mbox{\boldmath$\Theta$}}
\newcommand{\Phim}{\mbox{\boldmath$\Phi$}}
\newcommand{\Pim}{\mbox{\boldmath$\Pi$}}

\newcommand{\diag}{\mbox{diag}}
\newcommand{\tr}{\mbox{tr}}
\newcommand{\card}{\mbox{card}}
\newcommand{\cov}{\mbox{cov}}
\newcommand{\sign}{\mbox{sign}}
\newcommand{\var}{\mbox{var}}
\newcommand{\st}{\mbox{s.t.}}
\newcommand{\rank}{\mbox{rank}}
\newcommand{\argmin}{\mbox{argmin}}
\newcommand{\argmax}{\mbox{argmax}}

\newcommand{\Am}{\textbf{A}}
\newcommand{\Bm}{\textbf{B}}
\newcommand{\Cm}{\textbf{C}}
\newcommand{\Dm}{\textbf{D}}
\newcommand{\Em}{\textbf{E}}
\newcommand{\Fm}{\textbf{F}}
\newcommand{\Gm}{\textbf{G}}
\newcommand{\Hm}{\textbf{H}}
\newcommand{\Imat}{\textbf{I}}
\newcommand{\Jm}{\textbf{J}}
\newcommand{\Km}{\textbf{K}}
\newcommand{\Lm}{\textbf{L}}
\newcommand{\Mm}{\textbf{M}}
\newcommand{\Nm}{\textbf{N}}
\newcommand{\Om}{\textbf{O}}
\newcommand{\Pm}{\textbf{P}}
\newcommand{\Qm}{\textbf{Q}}
\newcommand{\Rm}{\textbf{R}}
\newcommand{\Sm}{\textbf{S}}
\newcommand{\Tm}{\textbf{T}}
\newcommand{\Um}{\textbf{U}}
\newcommand{\Vm}{\textbf{V}}
\newcommand{\Wm}{\textbf{W}}
\newcommand{\Xm}{\textbf{X}}
\newcommand{\Ym}{\textbf{Y}}
\newcommand{\Zm}{\textbf{Z}}

%Use regular expression: (\[a-z])([^a-zA-Z])  -> \1v\2  to change old style macros 
\graphicspath{{./Figures/}}

\title{Statistics and the Analysis of Data\\ Lecture 8: Confidence Intervals and Hypothesis Testing}
\author{Charanpal Dhanjal \\ \texttt{charanpal@gmail.com}} 
\institute{\'{E}cole des Ponts}
\date{14th January 2014}

\begin{document}

\frame{\titlepage}

\begin{frame}{Recap}  
\begin{itemize} 
 \item Statistical model - $(\mathcal{X}_n, \mathcal{P}_n)$, i.i.d. assumption 
 \item Estimators - bias and convergence 
 \item Maximum Likelihood Estimation 
\end{itemize} 
\end{frame}

\begin{frame}{Overview}
\begin{itemize} 
 \item How far is estimated value from true value? 
\item How do we test if hypothesis is correct? 
\end{itemize}
\end{frame}

\begin{frame}{Confidence Interval} 
\begin{itemize} 
 \item Let $X_1, \ldots, X_n$ be a set of observations from distribution $P_{\theta^*}$ with $\theta^* \in \Theta$. Then a \emph{confidence interval} for a set  $\hat{\theta} \subset \Theta$, in which $\hat{\theta}$ is estimated from the observations, takes the form  
\begin{displaymath} 
 P(\theta^* \in \hat{\theta}) > 1 - \alpha \quad \forall \theta^* \in \Theta 
\end{displaymath}
 where $\alpha \in [0, 1]$ is the degree of uncertainty 
\item Typically the range is $[\bar{\theta} - \delta, \bar{\theta} + \delta]$   
\end{itemize}
\end{frame}

\begin{frame}{Asymptotic Confidence}  
\begin{itemize} 
 \item In the case that $n \rightarrow \infty$ we have an \emph{asymptotic confidence interval} 
\end{itemize}
 \begin{displaymath} 
 \lim_{n \rightarrow \infty} P(\theta^* \in \hat{\theta}) > 1 - \alpha \quad \forall \theta^* \in \Theta 
\end{displaymath}
\end{frame}

\begin{frame}{An Example} 
\begin{itemize}
 \item We take a sample of 1000 adults from 60,000,000 adults in France and measure their heights in cm. The average height is 180cm with a standard deviation of 30cm. 
\item The 95\% confidence interval ($\alpha = 0.05$) is $180 \pm 1.86$.  
\item Steps: 
\begin{itemize}
\item Estimate mean 
\item Fix $\alpha$ 
\item Estimate interval of mean  
\end{itemize} 
\end{itemize}
\end{frame}

\begin{frame}{Asymmetric Range} 
\begin{itemize} 
 \item It might be that we have asymmetric range $[\bar{\theta} - \delta, \bar{\theta} + \delta']$
 \item One then defines 
\begin{displaymath} 
 P(\bar{\theta} - \theta^* < -\delta  ) \leq \frac{\alpha}{2} \quad \mbox{and} \quad P(\bar{\theta} - \theta^* > \delta') \leq \frac{\alpha}{2}
\end{displaymath}
\end{itemize}
\end{frame}
 
\begin{frame}{A Reminder: Air Quality Example} 
\begin{itemize}
 \item Want to find the distribution governing the ATMO index in Paris which rates air quality from 1 (best) to 10 (worst). 
 \item Look when index passes 8, and sample $n$ days in order to get some measurements. 
 \item The observations are denoted $X_1, \ldots, X_n$ and each case is 0 if the AMTO index is less than or equal to 8, otherwise 1. Assuming the random variable is Bernoulli one, we can write 
\begin{displaymath}
P(X = 0) = p^* \quad P(X = 1) = 1- p^*  
\end{displaymath}
for some $p^* \in (0, 1)$. 
\end{itemize}
\end{frame}

 \begin{frame}{The Question}  
 \begin{itemize} 
  \item We know that $\bar{\theta}^{ML} = \bar{X}$ 
  \item What is the range of $\theta^*$ such that the probability of it falling outside the range is 0.05? 
 \end{itemize}
 \end{frame}

\begin{frame}{A Useful Result} 
Let $X$ be a random variable with finite expected value $\mu$ and finite non-zero variance $\sigma^2$. Then for any real number $k > 0$, 
\begin{displaymath} 
 P(|X - \mu| > k \sigma) \leq \frac{1}{k^2},
\end{displaymath}
known as \emph{Chebyshev's inequality}. 
\end{frame}

\begin{frame}{Air Quality Confidence Interval}  
From  Chebyshev we fix $\delta = k\sigma \Rightarrow \frac{1}{k^2} = \frac{\sigma^2}{\delta^2}$ and 
\begin{eqnarray*} 
 P(|\bar{X} - \theta^*| > \delta) &\leq& \frac{\var(\bar{X})}{\delta^2} \\
&=& \frac{\var(\sum_i X_i)}{n^2\delta^2} \\ 
&=& \frac{\theta^*(1 - \theta^*)}{n\delta^2}  \\ 
&=& \frac{1}{4n\delta^2}  
\end{eqnarray*}
where in the last step $ab \leq (a+b)^2/4$. To complete example fix $\alpha = 0.05$ and solve to get $\delta = \sqrt{5/n}$. 
\end{frame}

\begin{frame}{Cups Example I}  
\begin{itemize} 
 \item Vending machine dispenses coffee with amount denoted by normal random variable $X$.  
 \item Assumed mean $\mu = 250$g and s.d. $\sigma = 2.5$g. 
 \item To calibrate machine sample 25 cups with sample mean $\bar{X} = 250.2$
 \item The \emph{standard error} (standard deviation of the error in the sample mean relative to the true mean) is defined as $\sigma_{\bar{X}} = \sigma/\sqrt{n} = 0.5$ 
 \item What is the error in our estimate of the population mean? 
\end{itemize}
\end{frame}

\begin{frame}{Cups Example II}  
\begin{itemize} 
 \item To get a standard normal variable we use $Z = \frac{\bar{X} - \mu}{\sigma_{\bar{X}}/\sqrt{n}}$ 
 \item We know that $P(-1.96 < Z < 1.96) = 0.95$ 
 \begin{itemize}
 \item $P(-1.96 < \frac{\bar{X} - \mu}{\sigma_{\bar{X}}/\sqrt{n}} < 1.96) = 0.95$ 
 \item $P(\bar{X} - 1.96 \sigma_{\bar{X}}/\sqrt{n}  < \mu < \bar{X} + 1.96 \sigma_{\bar{X}}/\sqrt{n}) = 0.95$ 
\end{itemize}
 \item Confidence interval: $\bar{X} \pm 1.96 \sigma_{\bar{X}}/\sqrt{n} = 250.2 \pm 0.98$ 
 \begin{itemize} 
 \item $\mu = 250$ within interval so no need for recalibration
 \end{itemize}
\end{itemize}
\end{frame}

\begin{frame}{Confidence in a Normal Distribution}  
\begin{itemize} 
 \item How did we know $P(-1.96 < Z < 1.96) = 0.95$? 
 \item We use a \emph{standard normal table} or \emph{Z table}
 \begin{itemize}
 \item Shows the cumulative distribution function for the standard normal distribution 
 \end{itemize} 
\end{itemize}
\end{frame}

\begin{frame}{Standard Normal Table}
\begin{itemize} 
 \item The elements in the table give the value of $q$ such that $\Phi(q) = 1 - z$ 
 \begin{itemize}
 \item $\Phi(q) = P(X \leq q)$ is the CDF for the standard normal distribution
 \item Implies $P(X > q) = z$
 \end{itemize} 
 \item So to get a $0.95$ confidence, set  $z = 0.025 \Rightarrow q = 1.96$ 
\end{itemize}
\end{frame}


\begin{frame}{Exercise} 
\begin{itemize} 
 \item For a particular class, the mean score in the exam is 80 with a standard deviation of 5. 
 \begin{itemize}
 \item What is the probability that a student scores an 82 or less? 
 \item What is the minimum number of students required so that we expect at least one to score above 90? 
 \item What is the probability that a student scores between 78 and 88?
 \item What is the probability that an average of three scores is 82 or less?
 \end{itemize} 
\end{itemize}
\end{frame}

\begin{frame}{Hypothesis Testing}  
\begin{itemize} 
\item Choice between $H_0: \theta^* \in \Theta_0$ and $H_1: \theta^* \in \Theta \setminus \Theta_0$
\item $H_0$ is the \emph{null hypothesis} and $H_1$ is the \emph{alternative hypothesis} 
\item Can we reject null hypothesis? 
\item Examples 
\begin{itemize} 
\item Is the defendant innocent or guilty? 
\item Is homoeopathy more than just a placebo? 
\item Do bumper stickers reflect car owner behaviour? 
\end{itemize}
\end{itemize}
\end{frame}

\begin{frame}{The Critical Region} 
\begin{itemize} 
 \item The \emph{critical} or \emph{regret region} is the subspace of $\mathbb{R}^n$ denoted $R_n$ which contains observations that support the null hypothesis.  
 \begin{itemize}
 \item We reject $H_0$ if $\{x_1, \ldots, x_n\} \in R_n$ 
 \item We do not reject $H_0$ if $\{x_1, \ldots, x_n\} \notin R_n$ 
 \end{itemize} 
\end{itemize}
\end{frame}

\begin{frame}{Error Types}  
\begin{itemize} 
 \item A \emph{Type I error} occurs when one rejects null hypothesis when true
 \begin{itemize}
 \item The risk of a type I error is $\alpha(R_n) = sup_{\theta^* \in \Theta_0} P(\{X_1, \ldots, X_n\} \in R_n)$
 \end{itemize} 
 \item A \emph{Type II error} occurs when one fails to reject false null hypothesis
  \begin{itemize}
 \item Risk of type II error is $\beta(R_n) = sup_{\theta^* \notin \Theta_0} P(\{X_1, \ldots, X_n\} \notin R_n)$
 \end{itemize} 
 \item Probability of not committing Type II error is the \emph{power} of test
 
\end{itemize}
\end{frame}

\begin{frame}{Test Quality}
\begin{itemize} 
 \item An ideal test has $\alpha(R_n) = \beta(R_n) = 0$
 \item More realistically, fix $\alpha \in (0, 1)$, then a test of \emph{level} $\alpha$ has $\alpha(R_n) \leq \alpha$ 
 \item One says $R_n$ has \emph{asymptotic level} $\alpha$ if $\lim_{n \rightarrow \infty} \alpha(R_n) \leq \alpha$
\end{itemize} 
\end{frame}

\begin{frame}{$p$-values} 
\begin{itemize} 
 \item Statistical tests $S_n$ can be written 
 \begin{displaymath} 
  R_{n, \alpha} = \{\{x_1, \ldots, x_n\}: S_n(x_1, \ldots, x_n) \geq C_\alpha\}
 \end{displaymath}
  where $C_\alpha \in \mathbb{R}$ is test criterion 
 \item Consider $\sup_{\theta^* \in \Theta_0} P(\{X_1, \ldots, X_n\} \in R_{n, \alpha}) = \alpha$
  \begin{itemize}
  \item $R_{n, \alpha}$ increases as $\alpha$ increases 
  \item There is a threshold $\alpha^*$ such that $R_{n, \alpha^*}$ contains the observations and $R_{n, \alpha}$ with $\alpha < \alpha^*$ do not 
  \item This is the $p$-value of $R_{n, \alpha}$ 
  \end{itemize} 
 \item In other words, the $p$-value is smallest $\alpha^*$ for which $R_{n, \alpha^*}$ rejects null hypothesis  
\end{itemize}
\end{frame}

\begin{frame}{Example: Binomial Distribution I}  
\begin{itemize} 
 \item Observations $X_1, \ldots, X_n \sim \mathcal{B}(\theta^*)$ 
 \item Null hypothesis $H_0: \theta^* = \theta_0$ and alternative $H_1: \theta^* \neq \theta_0$ with $\theta_0 = 0.1$ 
 \item To estimate $\theta^*$ we use $\bar{X} = \frac{1}{n}\sum_i X_i$ 
\end{itemize} 
\end{frame}

\begin{frame}{Example: Binomial Distribution II} 
\begin{itemize} 
\item Variance of Binomial distribution: $\var(X) = \theta^*(1- \theta)$
\item Using central limit theorem 
\begin{displaymath} 
 \frac{\sqrt{n}(\bar{X} - \theta^*)}{\sqrt{\theta^*(1- \theta^*)}} \xrightarrow[n \rightarrow \infty]{\mathcal{L}} \mathcal{N}(0, 1)
\end{displaymath}
\item Therefore we define $T(\bar{X}, \theta) = \frac{\sqrt{n}(\bar{X} - \theta)}{\sqrt{\theta(1- \theta)}}$ and $R_n = \{\{X_1, \ldots, X_n\}: |T(\bar{X}, \theta_0)| > b\}$
\item Normally we fix $\alpha$ and find $\lim_{n \rightarrow \infty} P(|T(\bar{X}, \theta_0)| > b) \leq \alpha$ 
\end{itemize}
\end{frame}

\begin{frame}{Example: Binomial Distribution III} 
\begin{itemize} 
 \item We know that $\lim_{n \rightarrow \infty} P(|T(\bar{X}, \theta_0)| > b) $ is the standard normal distribution
 \begin{itemize}
 \item Can find $b$ 
 \end{itemize} 
 \item Reject $H_0$ if 
 \begin{displaymath} 
 \frac{\sqrt{n}(\bar{X} - \theta)}{\sqrt{\theta(1- \theta)}} > b \Leftrightarrow |\bar{X} - 0.1| < \frac{0.3b}{\sqrt{n}} 
 \end{displaymath}
\item We do not reject $H_0$ if the above is not satisfied 
\end{itemize}
\end{frame}


\begin{frame}{Summary}
\begin{itemize} 
 \item Confidence interval  $P(\theta^* \in \hat{\theta}) > 1 - \alpha \quad \forall \theta^* \in \Theta $
 \begin{itemize}
 \item Computing intervals for normally distributed variables 
  \end{itemize}
 \item Hypothesis testing - choose between null and alternative hypotheses 
  \begin{itemize}
 \item $p$-values 
  \end{itemize}
\end{itemize}
\end{frame}


\end{document}