\documentclass{beamer}

\usepackage{beamerthemesplit}
\usepackage[utf8x]{inputenc}
\usepackage{pgf}
\usepackage{default}
\usepackage{url}
\usepackage{subfigure}
\usepackage{algorithmic} 
\usepackage{algorithm} 
\usepackage{amssymb}
\usepackage{graphicx}
\usepackage{booktabs}

\usetheme{Singapore}


%Define some commands for printing correct variables in math mode 
\newcommand{\av}{\textbf{a}}
\newcommand{\bv}{\textbf{b}}
\newcommand{\cv}{\textbf{c}}
\newcommand{\dv}{\textbf{d}}
\newcommand{\ev}{\textbf{e}}
\newcommand{\fv}{\textbf{f}}
\newcommand{\gv}{\textbf{g}}
\newcommand{\hv}{\textbf{h}}
\newcommand{\iv}{\textbf{i}}
\newcommand{\jv}{\textbf{j}}
\newcommand{\kv}{\textbf{k}}
\newcommand{\lv}{\textbf{l}}
\newcommand{\mv}{\textbf{m}}
\newcommand{\nv}{\textbf{n}}
\newcommand{\ov}{\textbf{o}}
\newcommand{\pv}{\textbf{p}}
\newcommand{\qv}{\textbf{q}}
\newcommand{\rv}{\textbf{r}}
\newcommand{\sv}{\textbf{s}}
\newcommand{\tv}{\textbf{t}}
\newcommand{\uv}{\textbf{u}}
\newcommand{\vv}{\textbf{v}}
\newcommand{\wv}{\textbf{w}}
\newcommand{\xv}{\textbf{x}}
\newcommand{\yv}{\textbf{y}}
\newcommand{\zv}{\textbf{z}}

\newcommand{\alphav}{\mbox{\boldmath$\alpha$}}
\newcommand{\betav}{\mbox{\boldmath$\beta$}}
\newcommand{\gammav}{\mbox{\boldmath$\gamma$}}
\newcommand{\xiv }{\mbox{\boldmath$\xi$}}
\newcommand{\muv}{\mbox{\boldmath$\mu$}}
\newcommand{\tauv}{\mbox{\boldmath$\tau$}}
\newcommand{\Omegam}{\mbox{\boldmath$\Omega$}}
\newcommand{\Lambdam}{\mbox{\boldmath$\Lambda$}}
\newcommand{\Sigmam}{\mbox{\boldmath$\Sigma$}}
\newcommand{\Gammam}{\mbox{\boldmath$\Gamma$}}
\newcommand{\Deltam}{\mbox{\boldmath$\Delta$}}
\newcommand{\Thetam}{\mbox{\boldmath$\Theta$}}
\newcommand{\Phim}{\mbox{\boldmath$\Phi$}}
\newcommand{\Pim}{\mbox{\boldmath$\Pi$}}

\newcommand{\diag}{\mbox{diag}}
\newcommand{\tr}{\mbox{tr}}
\newcommand{\card}{\mbox{card}}
\newcommand{\cov}{\mbox{cov}}
\newcommand{\sign}{\mbox{sign}}
\newcommand{\var}{\mbox{var}}
\newcommand{\st}{\mbox{s.t.}}
\newcommand{\rank}{\mbox{rank}}
\newcommand{\argmin}{\mbox{argmin}}
\newcommand{\argmax}{\mbox{argmax}}

\newcommand{\Am}{\textbf{A}}
\newcommand{\Bm}{\textbf{B}}
\newcommand{\Cm}{\textbf{C}}
\newcommand{\Dm}{\textbf{D}}
\newcommand{\Em}{\textbf{E}}
\newcommand{\Fm}{\textbf{F}}
\newcommand{\Gm}{\textbf{G}}
\newcommand{\Hm}{\textbf{H}}
\newcommand{\Imat}{\textbf{I}}
\newcommand{\Jm}{\textbf{J}}
\newcommand{\Km}{\textbf{K}}
\newcommand{\Lm}{\textbf{L}}
\newcommand{\Mm}{\textbf{M}}
\newcommand{\Nm}{\textbf{N}}
\newcommand{\Om}{\textbf{O}}
\newcommand{\Pm}{\textbf{P}}
\newcommand{\Qm}{\textbf{Q}}
\newcommand{\Rm}{\textbf{R}}
\newcommand{\Sm}{\textbf{S}}
\newcommand{\Tm}{\textbf{T}}
\newcommand{\Um}{\textbf{U}}
\newcommand{\Vm}{\textbf{V}}
\newcommand{\Wm}{\textbf{W}}
\newcommand{\Xm}{\textbf{X}}
\newcommand{\Ym}{\textbf{Y}}
\newcommand{\Zm}{\textbf{Z}}

%Use regular expression: (\[a-z])([^a-zA-Z])  -> \1v\2  to change old style macros 
\graphicspath{{./Figures/}}

\title{Statistics and the Analysis of Data\\ Lecture 5: Probability Distributions Part II}
\author{Charanpal Dhanjal \\ \texttt{charanpal@gmail.com}} 
\institute{\'{E}cole des Ponts}
\date{10th December 2013}

\begin{document}

\frame{\titlepage}


\begin{frame}{Recap}  
\begin{itemize} 
\item Discrete and continuous random variables map sample space to real number 
\item Expectation, variance 
\item Discrete distributions - Bernoulli, Binomial, Geometric, Poisson
\end{itemize}
\end{frame}

\begin{frame}{Outline} 
\begin{itemize} 
 \item Continuous distributions - uniform, exponential and Gaussian
 \item Random vectors 
 \item Multivariate normal distribution 
\end{itemize}
\end{frame}

\begin{frame}{Uniform Distribution}  
 \begin{itemize} 
\item Have already seen the discrete \emph{uniform distribution} e.g. when rolling a fair die. 
\item Taken to the limit, for an interval $[a, b]$ 
\begin{displaymath} 
 p(x) = \frac{1}{b-a}
\end{displaymath}
\item Denoted by $X \sim \mathcal{U}(a, b)$
\item Note that  $X \sim \mathcal{U}(a, b)$ implies $\frac{X - a}{b - a} \sim \mathcal{U}(0, 1)$
\item Expectation is $(a+b)/2$ and variance is $(b-a)^2/12$ 
\end{itemize}
\end{frame}



\begin{frame}{Exponential Distribution} 
\begin{itemize} 
 \item Buses arrive at an average rate of once every 10 minutes. Assuming arrival events are independent what is probability that two buses arrive 20 minutes apart? 
 \item We can model this type of event using the \emph{exponential distribution} (notation $X \sim \mathcal{E}(\theta)$)
 \begin{displaymath}
  p(x) = \left\{ \begin{array}{l l}\frac{1}{\theta}  \exp^{-x/\theta} & x \geq 0 \\ 0 & x < 0 \end{array} \right. 
 \end{displaymath}
  where $\theta$ is a scale parameter. 
\end{itemize}
\end{frame}

\begin{frame}{Exponential Distribution Properties}
\begin{itemize} 
 \item The exponential distribution is memoryless 
 \begin{displaymath}
  P(X > t + s | X > t) = P(X > s)
 \end{displaymath}
\item If $X \sim \mathcal{E}(\theta)$ then $X/\theta \sim \mathcal{E}(1)$
\item The expectation is $\theta$ and variance is $\theta^2$ 
\end{itemize}
\end{frame}

\begin{frame}{Exercise} 
\begin{itemize} 
 \item Show that the variance of the uniform distribution is $(b-a)^2/12$. 
 \item Write down an expression for the cumulative distribution function of a uniform random variable. The cumulative distribution function is defined as 
\begin{displaymath} 
 P(X < \gamma) = \int_{-\infty}^\gamma p(x) dx 
\end{displaymath}
\end{itemize}
\end{frame}

\begin{frame}{Normal Distribution}  
\begin{itemize} 
 \item The \emph{normal} or \emph{Gaussian distribution} is common in the real world e.g. test scores of students, heights of adults, speed of pedestrians 
\item $X$ is normally distributed (denoted $\mathcal{N}(\mu, \sigma^2)$) if it has a probability density function 
\begin{displaymath} 
 p(x) = \frac{1}{\sqrt{2\pi\sigma^2}}\exp^{-(x-\mu)^2/2\sigma^2} 
\end{displaymath}
where $\mu$ is the mean and $\sigma^2$ is the variance
\item $\mathcal{N}(0, 1)$ is known as the \emph{standard normal distribution} and if $X \sim \mathcal{N}(\mu, \sigma^2)$ then $\frac{X-\mu}{\sigma}$ is standard 
\end{itemize}
\end{frame}

\begin{frame}{Random Vectors}
\begin{itemize} 
 \item A \emph{random vector} $\xv = [\xv_1, \ldots, \xv_n]^T$ is one composed of random variables $\xv_i$ 
 \item Properties 
 \begin{itemize}
 \item If $\xv_i$ are discrete then $\xv$ is discrete 
 \item We say that $\xv$ is continuous if there exists $p: \mathbb{R}^n \rightarrow [0, \infty)$ such that $P(\xv \in A) = \int_A p(\xv) d \xv $ for all $A$
 \item If all $\xv_i$ are continuous then $\xv$ is not necessarily continuous
 \item If all $\xv_i$ are continuous and independent then $\xv$ is continuous and $p(\xv_1, \ldots, \xv_n) = p_{\xv_1}(\xv_1) \cdots p_{\xv_n}(\xv_n)$ 
 \end{itemize} 
 \item The expectation of a vector is the expectation of its elements  $\mathbb{E}[\xv] = [\mathbb{E}[\xv_1], \ldots, \mathbb{E}[\xv_n]]^T$ 
\end{itemize} 
\end{frame}

\begin{frame}{Covariance}  
\begin{itemize}
 \item The covariance of two random variables $X$ and $Y$ is $\cov(X, Y) = \mathbb{E}[(X - \mathbb{E}[X])(Y - \mathbb{E}[Y])]$ 
 \item Alternatively it can be written as $\cov(X, Y) = \mathbb{E}[XY] - \mathbb{E}[X]\mathbb{E}[Y]$
 \item Properties 
 \begin{itemize}
 \item $\cov(X, X) = \var(X)$ 
 \item If $X, Y$ are independent $\cov(X, Y) = 0$ 
 \item $\var(X + Y) = \var(X) + \var(Y) + 2\cov(X, Y)$ 
 \end{itemize} 
\end{itemize}
\end{frame}

\begin{frame}{Covariance Matrices} 
\begin{itemize} 
 \item A covariance matrix $\Cm \in \mathbb{R}^{n \times n}$ models the covariances between all pairs of elements of $\xv$
 \item In other words, $\Cm_{ij} = \cov(\xv_i, \xv_j)$
 \item Properties 
 \begin{itemize}
 \item $\Cm$ is symmetric and positive semi-definite (its eigenvalues are nonnegative) 
 \item If all $\xv_i$ are independent then $\Cm$ is diagonal 
 \item The converse is in general false: if $\Cm$ is diagonal this does not always imply $\xv_i$'s are independent 
 \item We can write $\Cm = \mathbb{E}[\xv\xv^T] - \mathbb{E}[\xv]\mathbb{E}[\xv^T]$
 \end{itemize} 
\end{itemize}
\end{frame}

\begin{frame}{Exercise}  
\begin{itemize} 
 \item Show that $\cov(X, Y) = \mathbb{E}[XY] - \mathbb{E}[X]\mathbb{E}[Y]$
 \item Show that $\Cm$ is symmetric positive semi-definite 
\end{itemize} 
\end{frame}

\begin{frame}{The Multivariate Gaussian}  
\begin{itemize} 
 \item Let $\muv \in \mathbb{R}^n$ be a vector of means, and $\Cm$ be a positive definite covariance matrix. 
 \item We can say that $\xv$ follows a multivariate Gaussian distribution, denoted $\mathcal{N}_n(\muv, C)$, if density function is 
 \begin{displaymath} 
  p(\xv) = \frac{1}{\sqrt{(2\pi)^n \det(\Cm)}} e^{-\frac{1}{2}(\xv - \muv)^T\Cm^{-1}(\xv - \muv)}  
 \end{displaymath}
 $\det$ is the \emph{determinant} and tells us about the matrix of coefficients of a system of linear equations
 \item Note that $\mathbb{E}[\xv] = \muv$ and $\var(\xv) = \Cm$  
 \item If $\xv_1, \ldots, \xv_n$ are independent Gaussian variables then $\xv$ is Gaussian 
\end{itemize}
\end{frame}

\begin{frame}{Exercise}  
\begin{itemize} 
 \item Show that the expectation of the standard normal distribution is zero 
\end{itemize}
\end{frame}


\begin{frame}{Summary}  
\begin{itemize} 
 \item Some continuous probability distributions - exponential, uniform, normal 
 \item Random vectors, covariance matrices
\end{itemize}
\end{frame}

\end{document}