\begin{frame}{Uniform Distribution}  
 \begin{itemize} 
  \item We have already seem the discrete uniform distribution e.g. when rolling a fair die. 
\item Taken to the limit, we have for an interval $[a, b]$ 
\begin{displaymath} 
 p(x) = \frac{1}{b-a}
\end{displaymath}
\item Denoted by $X \sim \mathcal{U}(a, b)$
\item Note that  $X \sim \mathcal{U}(a, b)$ implies $\frac{X - a}{b - a} \sim \mathcal{U}(0, 1)$
\end{itemize}
\end{frame}

\begin{frame}{Exponential Distribution} 
\begin{itemize} 
 \item For the bus example what is probability that two buses arrive 20 minutes apart? 
 \item We can model this type of event using the \emph{Exponential distribution} with notation $X \sim \mathcal{E}(\theta)$
 \begin{displaymath}
  p(x) = \frac{1}{\theta} \left\{ \begin{array}{l l} \exp^{-x/\theta} & x \geq 0 \\ 0 & x < 0 \end{array} \right. 
 \end{displaymath}
  where $\theta$ is a scale parameter. 
\end{itemize}
\end{frame}

\begin{frame}{Exponential Distribution Properties}
\begin{itemize} 
 \item The exponential distribution is memoryless 
 \begin{displaymath}
  P(X > T + s | X > t) = P(X > s)
 \end{displaymath}
\item If $X \sim \mathcal{E}(\theta)$ then $X\theta \sim \mathcal{E}(1)$
\end{itemize}
\end{frame}

\begin{frame}{Normal Distribution}  
 
\end{frame}