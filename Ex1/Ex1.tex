\documentclass[a4paper,10pt]{article}
\usepackage[utf8]{inputenc}
\usepackage{url}
\usepackage{pgf}
\usepackage{listings}
\usepackage{amsfonts,enumerate,amsmath,amssymb,enumitem,graphicx}
\usepackage[margin=1.5in]{geometry}

%opening
\title{Statistics and the Analysis of Data\\ Exercise Sheet 1}
\author{Charanpal Dhanjal}
%\institute{\'{E}cole des Ponts}

\definecolor{light-gray}{gray}{0.95}

\begin{document}
\lstset{language=R, showspaces=false, showstringspaces=false, backgroundcolor=\color{light-gray}}  
\date{20th January 2014}
\maketitle

\section*{Exercise 1}

Note that the standard normal table is provided via the ``polycopy'' and also the handout from the previous lecture. We will denote $\Phi(q)= \mathbb{P}(Z\le q)$ as the cumulative distribution function for a standard Gaussian random variable $Z$. 

\begin{enumerate} 
 

\item  Show that
\begin{equation}\label{eq:quantile}
\mathbb{P} (|Z|\le q) = 2 \mathbb{P} (Z\le q) -1.
\end{equation}
and then demonstrate
\begin{eqnarray*}
\mathbb{P}(|Z|\le q ) = 0.95 &\Leftrightarrow & \mathbb{P} ( Z\le q) = 0.975.\\
\mathbb{P}(|Z|\le q ) = 0.99 &\Leftrightarrow & \mathbb{P} ( Z\le q) = 0.995.\\
\end{eqnarray*}
\end{enumerate}

The compression force (the force a material requires to squash or compact it) of a particular type of concrete is modelled using a Gaussian random variable with mean $\mu$ and variance $\sigma^2$ in \emph{pounds per square inch} (psi). For questions 2 to 5, we assume the variance is known and equal to 1000. In a sample $(x_1,\cdots, x_n)$ of 12 measurements, we observe an empirical mean of 3250 psi. Recall that the mean $\frac 1n \sum_{i=1}^n x_i$ follows a Gaussian distribution with mean $\mu$ and variance $\frac{\sigma^2}n$ and $\frac{\sqrt{n}}{\sigma} \left( \frac 1n \sum_{i=1}^n x_i - \mu\right)$ follows the standard Gaussian distribution. Be sure to use the corrected version of the confidence interval slides: \url{http://perso.telecom-paristech.fr/~cdhanjal/Lecture8.pdf}

\begin{enumerate}
\setcounter{enumi}{1}
\item Give a confidence interval of level 0.95 for $\mu$. Justify your answer. 
\item Give a confidence interval of level 0.99 for $\mu$. Compare the size of the this interval with the previous one. 
\item If, using the same sample, we give a confidence interval with width 30 psi, what is the associated confidence level? 
\item We wish to estimate $\mu$ with a confidence interval of 30 psi, and a confidence level of 0.95. What is the minimum sample size we require? 
\end{enumerate}

\section*{Exercise 2}

Note that  
$$\sum_{k=0}^\infty \frac{\lambda^k}{k!} = e^\lambda\qquad\textrm{et}\qquad \sum_{k=0}^\infty k^2 \frac{ \lambda^k}{k!}=e^{\lambda}(\lambda^2 + \lambda).
$$ 
Let $(X_1,\cdots, X_n)$ be a sample of Poisson random variables with parameter $\lambda^*$ unknown:
$$
\mathbb{P}(X_i=k) = e^{-\lambda^*} \frac{(\lambda^*)^k}{k!}\ ,\quad k\ge 0.
$$
\begin{enumerate}
\item Show that $\mathbb{P}$ is a probability distribution and calculate $\mathbb{E}[X]$.
\item Calculate the likelihood of the sample. 
\item Deduce the estimator $\hat{\lambda}_n$ which is the maximum likelihood estimation of $\lambda^*$.
\item Is this estimator unbiased? 
\item Calculate the quadratic risk of the estimator:
$$
\mathbb{E}_{\lambda^*}\left[ \left| \hat \lambda_n -\lambda^*\right|^2\right].
$$
\item Is the estimator $\hat \lambda_n$  asymptotically normal?
\end{enumerate}

\section*{Submission}

Please submit the answers to these exercises before midnight CET on 28/1/14 using the heading ``Statistics and the Analysis of Data Exercise Sheet 1''. Remember to write your name at the top of your answer sheet. 

\end{document}
