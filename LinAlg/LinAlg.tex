\documentclass{beamer}

\usepackage{beamerthemesplit}
\usepackage[utf8x]{inputenc}
\usepackage{pgf}
\usepackage{default}
\usepackage{url}
\usepackage{subfigure}
\usepackage{algorithmic} 
\usepackage{algorithm} 
\usepackage{amssymb}
\usepackage{graphicx}
\usepackage{booktabs}

\usetheme{Singapore}


%Define some commands for printing correct variables in math mode 
\newcommand{\av}{\textbf{a}}
\newcommand{\bv}{\textbf{b}}
\newcommand{\cv}{\textbf{c}}
\newcommand{\dv}{\textbf{d}}
\newcommand{\ev}{\textbf{e}}
\newcommand{\fv}{\textbf{f}}
\newcommand{\gv}{\textbf{g}}
\newcommand{\hv}{\textbf{h}}
\newcommand{\iv}{\textbf{i}}
\newcommand{\jv}{\textbf{j}}
\newcommand{\kv}{\textbf{k}}
\newcommand{\lv}{\textbf{l}}
\newcommand{\mv}{\textbf{m}}
\newcommand{\nv}{\textbf{n}}
\newcommand{\ov}{\textbf{o}}
\newcommand{\pv}{\textbf{p}}
\newcommand{\qv}{\textbf{q}}
\newcommand{\rv}{\textbf{r}}
\newcommand{\sv}{\textbf{s}}
\newcommand{\tv}{\textbf{t}}
\newcommand{\uv}{\textbf{u}}
\newcommand{\vv}{\textbf{v}}
\newcommand{\wv}{\textbf{w}}
\newcommand{\xv}{\textbf{x}}
\newcommand{\yv}{\textbf{y}}
\newcommand{\zv}{\textbf{z}}

\newcommand{\alphav}{\mbox{\boldmath$\alpha$}}
\newcommand{\betav}{\mbox{\boldmath$\beta$}}
\newcommand{\gammav}{\mbox{\boldmath$\gamma$}}
\newcommand{\xiv }{\mbox{\boldmath$\xi$}}
\newcommand{\muv}{\mbox{\boldmath$\mu$}}
\newcommand{\tauv}{\mbox{\boldmath$\tau$}}
\newcommand{\Omegam}{\mbox{\boldmath$\Omega$}}
\newcommand{\Lambdam}{\mbox{\boldmath$\Lambda$}}
\newcommand{\Sigmam}{\mbox{\boldmath$\Sigma$}}
\newcommand{\Gammam}{\mbox{\boldmath$\Gamma$}}
\newcommand{\Deltam}{\mbox{\boldmath$\Delta$}}
\newcommand{\Thetam}{\mbox{\boldmath$\Theta$}}
\newcommand{\Phim}{\mbox{\boldmath$\Phi$}}
\newcommand{\Pim}{\mbox{\boldmath$\Pi$}}

\newcommand{\diag}{\mbox{diag}}
\newcommand{\tr}{\mbox{tr}}
\newcommand{\card}{\mbox{card}}
\newcommand{\cov}{\mbox{cov}}
\newcommand{\sign}{\mbox{sign}}
\newcommand{\var}{\mbox{var}}
\newcommand{\st}{\mbox{s.t.}}
\newcommand{\rank}{\mbox{rank}}
\newcommand{\argmin}{\mbox{argmin}}
\newcommand{\argmax}{\mbox{argmax}}

\newcommand{\Am}{\textbf{A}}
\newcommand{\Bm}{\textbf{B}}
\newcommand{\Cm}{\textbf{C}}
\newcommand{\Dm}{\textbf{D}}
\newcommand{\Em}{\textbf{E}}
\newcommand{\Fm}{\textbf{F}}
\newcommand{\Gm}{\textbf{G}}
\newcommand{\Hm}{\textbf{H}}
\newcommand{\Imat}{\textbf{I}}
\newcommand{\Jm}{\textbf{J}}
\newcommand{\Km}{\textbf{K}}
\newcommand{\Lm}{\textbf{L}}
\newcommand{\Mm}{\textbf{M}}
\newcommand{\Nm}{\textbf{N}}
\newcommand{\Om}{\textbf{O}}
\newcommand{\Pm}{\textbf{P}}
\newcommand{\Qm}{\textbf{Q}}
\newcommand{\Rm}{\textbf{R}}
\newcommand{\Sm}{\textbf{S}}
\newcommand{\Tm}{\textbf{T}}
\newcommand{\Um}{\textbf{U}}
\newcommand{\Vm}{\textbf{V}}
\newcommand{\Wm}{\textbf{W}}
\newcommand{\Xm}{\textbf{X}}
\newcommand{\Ym}{\textbf{Y}}
\newcommand{\Zm}{\textbf{Z}}

%Use regular expression: (\[a-z])([^a-zA-Z])  -> \1v\2  to change old style macros 
\graphicspath{{./Figures/}}

\title{Statistics and the Analysis of Data\\ A Barebones Primer on Linear Algebra}
\author{Charanpal Dhanjal \\ \texttt{charanpal@gmail.com}} 
\institute{\'{E}cole des Ponts}
\date{5th November 2013}

\begin{document}

\frame{\titlepage}

\begin{frame}{Introduction}
\begin{itemize} 
 \item What is linear algebra? 
 \begin{itemize}
 \item Branch of maths concerning vector spaces and linear mapping in these spaces. 
 \end{itemize} 
 \item Applications: 
 \begin{itemize}
 \item Computer graphics - 3D rendering, image processing  
 \item Web search - PageRank
 \item Image compression 
 \item Computational statistics 
 \item ... and lots of others 
 \end{itemize} 
\end{itemize}
\end{frame}

\begin{frame}{Overview} 
\begin{itemize}
 \item Vectors, matrices 
 \item Norms, inner products, projections  
 \item Eigenvalues, eigenvectors 
\end{itemize} 
\end{frame}

\begin{frame}{Vectors}
\begin{itemize} 
 \item A \emph{vector} represents a displacement, e.g. $\vv \in \mathbb{R}^d$ 
 \item We will consider vectors as a column array 
 \item The \emph{dot} or \emph{inner} product between two vectors is defined as $\langle \uv, \vv \rangle = \uv \cdot \vv = \uv^T\vv = \sum_{i=1}^d \uv_i\vv_i$ (T means transpose)
 
\end{itemize}
\end{frame}

\begin{frame}{Inner Products}
\begin{itemize}
  \item Some properties of the inner products
 \begin{itemize}
 \item $\langle \uv, \vv \rangle = \langle \vv, \uv \rangle$ commutativity 
 \item $\langle \uv, (\vv + \wv) \rangle = \langle \uv, \vv \rangle + \langle \uv, \wv \rangle$ distributivity 
 \item $c \langle \uv, \vv \rangle = \langle c \uv, \vv \rangle$, $c$ is a scalar 
 \item $ \langle \uv, \uv \rangle \geq 0$ and $ \langle \uv, \uv \rangle = 0$ iff $\uv = \textbf{0}$  
 \end{itemize}
\end{itemize}
\end{frame}


\begin{frame}{Vector Norm}  
\begin{itemize} 
 \item The \emph{Euclidean} or \emph{2-norm} of a vector is denoted $\|\uv\| =  \sqrt{\langle \uv, \uv \rangle} = \sqrt{\sum_{i=1}^d \uv_i^2}$
 \item It obeys the following properties	
 \begin{itemize}
 \item  $\|\uv\| = 0$ iff $\uv = \textbf{0}$ 
 \item $\|c\uv\| = |c|\|\uv\|$
 \end{itemize}
 \item The Cauchy-Swartz inequality
  \begin{displaymath} 
   |\langle \uv, \vv \rangle| \leq \|\uv\|\|\vv\|
  \end{displaymath}
\end{itemize}
\end{frame}

\begin{frame}{Vector Distances and Angles}  
\begin{itemize} 
 \item The \emph{distance} between two vectors is $d(\uv, \vv) = \|\uv - \vv\|$ 
 \item Angle between vectors 
 \begin{displaymath} 
  \cos(\theta) = \frac{\langle \uv, \vv \rangle}{\|\uv\|\|\vv\|}
 \end{displaymath}
 \item Two vectors are \emph{orthogonal} if their dot product is zero 
 \item A \emph{projection} of $\vv$ onto $\uv$ is given by $\frac{\langle \uv, \vv \rangle}{\langle \uv, \uv \rangle}\uv$   
\end{itemize}
\end{frame} 

\begin{frame}{Matrices} 
\begin{itemize}
 \item A \emph{matrix} is a rectangular array of numbers or alternatively a column-wise concatenation of vectors 
 \item When we say a matrix $\Am$ has size $m \times n$ this means $m$ rows and $n$ columns 
 \item Matrix types 
 \begin{itemize}
 \item \emph{Square} when $m = n$ 
 \item \emph{Symmetric} when $\Am^T = \Am$ (by transpose, we mean $\Am_{ij} = \Am_{ji}$ )
 \item \emph{Diagonal} when $\Am_{ij} = 0, i \neq j$ 
 \item \emph{Identity} is a diagonal matrix with $\Am_{ii} = 1, \forall i$ and written $\Imat$ 
 \end{itemize}
\end{itemize}
\end{frame}

\begin{frame}{Matrix Multiplication I} 
\begin{itemize} 
 \item Hadamard product: $(\Am \cdot \Bm)_{ij} = \Am_{ij} + \Bm_{ij}$ 
 \item Dot product $(\Am\Bm)_{ij} = \sum_{k=1}^n \Am_{ik}\Bm_{kj}$
 \item When we talk about matrix multiplication we usually mean the second case 
\end{itemize}
\end{frame}

\begin{frame}{Matrix Multiplication II}
\begin{itemize} 
  \item Properties 
 \begin{itemize}
 \item $(\Am\Bm)\Cm = \Am(\Bm\Cm)$ Associativity 
 \item $\Am(\Bm + \Cm) = \Am\Bm + \Am\Cm$ Left distributivity 
 \item $(\Am +\Bm) \Cm = \Am\Cm +\Bm\Cm$ Right distributivity 
 \item $\Am\Imat = \Imat\Am = \Am$ Multiplicative identity 
 \end{itemize} 
 \end{itemize}
\end{frame}


\begin{frame}{Matrix Transpose and Trace} 
 \begin{itemize} 
  \item Some properties of the transpose  
  \begin{itemize}
  \item $(\Am^T)^T = \Am$ 
  \item $(\Am + \Bm)^T = \Am^T = \Bm^T$ 
  \item $(\Am\Bm)^T = \Bm^T\Am^T$  
  \end{itemize} 
 \item The matrix \emph{trace} is the sum of the diagonal elements $\tr(\Am) = \sum_{i=1}^n \Am_{ii}$ 
 \begin{itemize} 
 \item Traces allow for cyclic permutations $\tr(\Am\Bm\Cm) = \tr(\Cm\Am\Bm) = \tr(\Bm\Cm\Am)$ 
 \end{itemize}
 \end{itemize}
\end{frame}

\begin{frame}{Exercise} 
Show the following properties are true
\begin{itemize}
 \item $\Am(\Bm + \Cm) = \Am\Bm + \Am\Cm$ Left distributivity 
 \item $(\Am +\Bm) \Cm = \Am\Cm +\Bm\Cm$ Right distributivity 
 \item Cyclic permutation: $\tr(\Am\Bm\Cm) = \tr(\Cm\Am\Bm) = \tr(\Bm\Cm\Am)$ 
\end{itemize}
\end{frame}

\begin{frame}{Eigenvalues and Eigenvectors}  
\begin{itemize} 
 \item Assume $\Am \in \mathbb{R}^{n \times n}$. A scalar $\lambda$ is an \emph{eigenvalue} of $\Am$ if there is a nonzero vector $\vv$ (known as a \emph{eigenvector}) such that $\Am\vv = \lambda\vv$
 \item In general we fix $\|\vv\| = 1$ 
 \item We can see that vectors $\Am\vv$ and $\vv$ are parallel, so $\vv$ is a eigenvector if and only if $\Am$ transforms $\vv$ into a parallel vector  
 \item If $\Am$ is symmetric we can write it as an \emph{eigen-decomposition} $\Am = \sum_{i=1}^n \lambda_i\vv_i\vv^T$ 
 \item The trace is the sum of eigenvalues $\tr(\Am) = \sum_{i=1}^n \lambda_i$
\end{itemize}
\end{frame}

\begin{frame}{What We Did Not Cover}  
\begin{itemize} 
 \item Systems of linear equations 
 \item Subspaces, span, basis, rank 
 \item Decompositions - SVD, QR, Cholesky, Schur, LU 
 \item Vector/matrix calculus 
\end{itemize}
\end{frame}



\end{document} 