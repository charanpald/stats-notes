\documentclass{beamer}

\usepackage{beamerthemesplit}
\usepackage[utf8x]{inputenc}
\usepackage{pgf}
\usepackage{default}
\usepackage{url}
\usepackage{subfigure}
\usepackage{algorithmic} 
\usepackage{algorithm} 
\usepackage{amssymb}
\usepackage{graphicx}
\usepackage{booktabs}
\usepackage{mathtools}

\usetheme{Singapore}


%Define some commands for printing correct variables in math mode 
\newcommand{\av}{\textbf{a}}
\newcommand{\bv}{\textbf{b}}
\newcommand{\cv}{\textbf{c}}
\newcommand{\dv}{\textbf{d}}
\newcommand{\ev}{\textbf{e}}
\newcommand{\fv}{\textbf{f}}
\newcommand{\gv}{\textbf{g}}
\newcommand{\hv}{\textbf{h}}
\newcommand{\iv}{\textbf{i}}
\newcommand{\jv}{\textbf{j}}
\newcommand{\kv}{\textbf{k}}
\newcommand{\lv}{\textbf{l}}
\newcommand{\mv}{\textbf{m}}
\newcommand{\nv}{\textbf{n}}
\newcommand{\ov}{\textbf{o}}
\newcommand{\pv}{\textbf{p}}
\newcommand{\qv}{\textbf{q}}
\newcommand{\rv}{\textbf{r}}
\newcommand{\sv}{\textbf{s}}
\newcommand{\tv}{\textbf{t}}
\newcommand{\uv}{\textbf{u}}
\newcommand{\vv}{\textbf{v}}
\newcommand{\wv}{\textbf{w}}
\newcommand{\xv}{\textbf{x}}
\newcommand{\yv}{\textbf{y}}
\newcommand{\zv}{\textbf{z}}

\newcommand{\alphav}{\mbox{\boldmath$\alpha$}}
\newcommand{\betav}{\mbox{\boldmath$\beta$}}
\newcommand{\gammav}{\mbox{\boldmath$\gamma$}}
\newcommand{\xiv }{\mbox{\boldmath$\xi$}}
\newcommand{\muv}{\mbox{\boldmath$\mu$}}
\newcommand{\tauv}{\mbox{\boldmath$\tau$}}
\newcommand{\Omegam}{\mbox{\boldmath$\Omega$}}
\newcommand{\Lambdam}{\mbox{\boldmath$\Lambda$}}
\newcommand{\Sigmam}{\mbox{\boldmath$\Sigma$}}
\newcommand{\Gammam}{\mbox{\boldmath$\Gamma$}}
\newcommand{\Deltam}{\mbox{\boldmath$\Delta$}}
\newcommand{\Thetam}{\mbox{\boldmath$\Theta$}}
\newcommand{\Phim}{\mbox{\boldmath$\Phi$}}
\newcommand{\Pim}{\mbox{\boldmath$\Pi$}}

\newcommand{\diag}{\mbox{diag}}
\newcommand{\tr}{\mbox{tr}}
\newcommand{\card}{\mbox{card}}
\newcommand{\cov}{\mbox{cov}}
\newcommand{\sign}{\mbox{sign}}
\newcommand{\var}{\mbox{var}}
\newcommand{\st}{\mbox{s.t.}}
\newcommand{\rank}{\mbox{rank}}
\newcommand{\argmin}{\mbox{argmin}}
\newcommand{\argmax}{\mbox{argmax}}

\newcommand{\Am}{\textbf{A}}
\newcommand{\Bm}{\textbf{B}}
\newcommand{\Cm}{\textbf{C}}
\newcommand{\Dm}{\textbf{D}}
\newcommand{\Em}{\textbf{E}}
\newcommand{\Fm}{\textbf{F}}
\newcommand{\Gm}{\textbf{G}}
\newcommand{\Hm}{\textbf{H}}
\newcommand{\Imat}{\textbf{I}}
\newcommand{\Jm}{\textbf{J}}
\newcommand{\Km}{\textbf{K}}
\newcommand{\Lm}{\textbf{L}}
\newcommand{\Mm}{\textbf{M}}
\newcommand{\Nm}{\textbf{N}}
\newcommand{\Om}{\textbf{O}}
\newcommand{\Pm}{\textbf{P}}
\newcommand{\Qm}{\textbf{Q}}
\newcommand{\Rm}{\textbf{R}}
\newcommand{\Sm}{\textbf{S}}
\newcommand{\Tm}{\textbf{T}}
\newcommand{\Um}{\textbf{U}}
\newcommand{\Vm}{\textbf{V}}
\newcommand{\Wm}{\textbf{W}}
\newcommand{\Xm}{\textbf{X}}
\newcommand{\Ym}{\textbf{Y}}
\newcommand{\Zm}{\textbf{Z}}

%Use regular expression: (\[a-z])([^a-zA-Z])  -> \1v\2  to change old style macros 
\graphicspath{{./Figures/}}

\title{Statistics and the Analysis of Data\\ Lecture 7: Parametric Estimation}
\author{Charanpal Dhanjal \\ \texttt{charanpal@gmail.com}} 
\institute{\'{E}cole des Ponts}
\date{7th January 2014}

\begin{document}

\frame{\titlepage}

\begin{frame}{Recap}  
\begin{itemize} 
\item Difference types of convergence - almost surely, mean, probability 
\item Strong law of large numbers 
\item Central Limit Theorem 
\end{itemize}
\end{frame}

\begin{frame}{Outline}  
\begin{itemize} 
 \item Parametric estimation 
 \item Bias, convergence
 \item Maximum likelihood estimation
\end{itemize}
\end{frame}

\begin{frame}{Introduction}
\begin{itemize}
 \item Have some observations of a random variable with an unknown distribution. Assume we can guess a set of distributions that might generate the observations. 
 \item How can we find the correct distribution? 
 \item For example, if we assume the variable is distributed using the Poisson distribution with rate $\theta$ 
 \item This is the central aim of \emph{parametric estimation}
\end{itemize}
\end{frame}

\begin{frame}{Some Problems} 
\begin{itemize} 
 \item Three problems to consider 
 \begin{itemize} 
 \item Estimate the unknown parameter 
 \item Determine a region in which we know the parameter lies with a certain probability 
 \item How does our estimation of the parameter improve as we gather more observations? 
 \end{itemize} 
\end{itemize}
\end{frame}

\begin{frame}{Example: Air Quality} 
\begin{itemize}
 \item Want to find the distribution governing the ATMO index in Paris which rates air quality from 1 (best) to 10 (worst). 
 \item Look when index passes 8, and sample $n$ days in order to get some measurements. 
 \item The observations are denoted $X_1, \ldots, X_n$ and each case is 0 if the AMTO index is less than or equal to 8, otherwise 1. Assuming the random variable is Bernoulli one, we can write 
\begin{displaymath}
P(X = 0) = p^* \quad P(X = 1) = 1- p^*  
\end{displaymath}
for some $p^* \in (0, 1)$. 
\end{itemize}
\end{frame}

\begin{frame}{Example 2: Wind Power} 
\begin{itemize} 
 \item To determine if wind generator is appropriate: measure wind speed $v$ on a sample of days. What is the probability that $v \geq 10$Km/h?
 \item The sample for $n$ days is denoted $v_1, v_2, \ldots, v_n$ and $v_i^2$ is modelled using an exponential distribution $\mathcal{E}(\theta)$ with parameter $\theta$: 
  \begin{displaymath}
  p(x) = \left\{ \begin{array}{l l}\frac{1}{\theta}  \exp^{-x/\theta} & x \geq 0 \\ 0 & x < 0 \end{array} \right. 
 \end{displaymath}
 \item Want to estimate the true value of $\theta$ from the data 
\end{itemize}
\end{frame}


\begin{frame}{Statistical Models} 
\begin{itemize} 
 \item We have a set of observations $X_1, \ldots, X_n$
 \item A \emph{statistical model} is tuple $(\mathcal{X}_n, \mathcal{P}_n)$ in which 
 \begin{itemize} 
 \item $\mathcal{X}_n$ is the state space 
  \item $\mathcal{P}_n$ a collection of distribution functions $\mathcal{P}_n = \{P_{n, \theta}, \theta \in \Theta\}$ where $\theta$ is a parameter and $n$ is the number of observations 
 \end{itemize}
  \item For example $\mathcal{X}_n = \mathbb{R}^n$, $\mathcal{P}_n$ could be a set of $n$ Gaussian distributions $\{\mathcal{N}(\mu_1, \sigma^2_1), \mathcal{N}(\mu_2, \sigma^2_2), \ldots, \mathcal{N}(\mu_n, \sigma^2_n)\}$ with $\mu_i \in \mathbb{R}$ and $\sigma_i > 0$ for all $i$ 
\end{itemize}
\end{frame}

\begin{frame}{Some Assumptions}
\begin{itemize} 
 \item We assume that the model represents the data 
 \item In general it is impossible to infer the distribution given the observations without assuming a certain structure 
 \item Assume the observations are i.i.d. which simplifies the distribution
 \begin{displaymath} 
  P_{n, \theta} = P_{\theta} \otimes \cdots \otimes  P_{\theta}  
 \end{displaymath}
\end{itemize}
\end{frame}

\begin{frame}{Example Of Non-i.i.d. Data}
\begin{itemize} 
 \item Samples can also be independent but not identically distributed, e.g. 
 \begin{displaymath} 
  X_i = f\left(\frac{i}{n} - \theta  \right) + \epsilon_i, \quad i = 1,\ldots, n
 \end{displaymath}
 where $\epsilon_i \sim \mathcal{N}(0, 1)$, $f$ is a fixed function and 
 \begin{displaymath} 
  P_{n, \theta} = P_{\theta}^1 \otimes \cdots \otimes  P_{\theta}^n  
 \end{displaymath}
  and $P^i_\theta$ is the distribution $\mathcal{N}(f(i/n - \theta), 1)$ 
\end{itemize}
\end{frame}


\begin{frame}{Estimators} 
\begin{itemize} 
\item Can write $X_1, \ldots, X_n \sim_{i.i.d.} P_{\theta^*}$ to mean we draw all observations from distribution $P$ with unknown $\theta^* \in \Theta$
\item Denote $S_{\theta^*,n}$ a particular sample of size $n$ drawn from $P_{\theta^*}$ 
\item We call $\theta^*$ the \emph{true value} of the parameter
\item An \emph{estimator} is a function $\rho: \mathbb{R}^n \rightarrow \Theta$ and $\bar{\theta}_n = \rho(S_{\theta^*,n})$. In a slight abuse of notation we can also write $\bar{\theta}_n(S_{\theta^*,n})$
\item Would like estimated value $\bar{\theta}_n$ as close as possible to $\theta^*$
\end{itemize}
\end{frame}

\begin{frame}{Examples of Estimators}  
\begin{itemize} 
 \item Suppose that $\mathcal{P}$ is the normal distribution $\{\mathcal{N}(\theta, 1), \theta \in \mathbb{R}\}$  
 \item The following are estimators of $\theta^*$ 
 \begin{itemize}
 \item $\bar{\theta}_n(S_{\theta^*,n}) = 1$
 \item $\bar{\theta}_n(S_{\theta^*,n}) = X_1$
 \item $\bar{\theta}_n(S_{\theta^*,n}) = \bar{X}$
 \end{itemize} 
 \item Which of these estimators are most appropriate? 
\end{itemize}
\end{frame}

\begin{frame}{Bias and Convergence}  
\begin{itemize} 
\item We can say that our estimated parameter is \emph{unbiased} if $\mathbb{E}[\rho(S_{\theta^*,n})] = \theta^*$ for all $\theta^* \in \Theta$
\item The estimator is \emph{consistent} or \emph{convergent} if 
\begin{displaymath} 
 \lim_{n \rightarrow \infty} P_{\theta^*}(|\bar{\theta}_n - \theta^*| > \epsilon) = 0), \quad \forall \epsilon > 0, \theta^* \in \Theta 
\end{displaymath}
\item In other words the estimated parameter $\bar{\theta}_n$ approaches the true value as $n$ increases 
\item Estimators are random variables and can also consider other types of convergence from the previous lecture
\end{itemize}
\end{frame}

\begin{frame}{Asymptotic Distribution} 
\begin{itemize} 
 \item How fast does convergence occur? 
 \item We can say that our estimated parameter $\bar{\theta}_n$ is asymptotically distributed with $P_{\theta^*}^\infty$ and rate $n^{-\gamma}$, $\gamma > 0$, if 
 \begin{displaymath}
  n^\gamma(\bar{\theta}_n - \theta^*) \xrightarrow[]{\mathcal{L}} P_{\theta^*}^\infty, \quad \forall  \theta^* \in \Theta, 
 \end{displaymath}
where $\xrightarrow[]{\mathcal{L}}$ means distribution convergence. 
\item In the special case $P_{\theta^*}^\infty$ is Gaussian $\mathcal{N}(0, \sigma^2_{\theta^*})$ we say $\bar{\theta}_n$ is \emph{asymptotically Normal} with rate $n^{-\gamma}$ and variance $\sigma^2_{\theta^*}$ 
\end{itemize}
\end{frame}

\begin{frame}{Maximum Likelihood Estimation} 
\begin{itemize} 
 \item One of the most commonly used methods for parameter estimation is \emph{Maximum Likelihood Estimation} (MLE) 
 \item The idea is to solve 
 \begin{displaymath}
  \bar{\theta}^{MV} = \argmax_{\theta \in \Theta} \; p_n(\theta; X_1, \ldots, X_n) 
 \end{displaymath}
\end{itemize}
\end{frame}

\begin{frame}{Example} 
\begin{itemize} 
 \item For the air quality example we can use the distributions $\{\mathcal{B}(\theta), \theta \in (0, 1)\}$ 
 \item For each $\theta$ we calculate $p_n(\theta; X_1, \ldots, X_n)$
 \item Note that $\mathcal{X} \in \{0, 1\}$ and that $p(\theta, X) = \theta^X(1-\theta)^{1-X}$  
 \item Follows that 
 \begin{eqnarray*} 
  p(\theta; X_1, \ldots, X_n) &=& \theta^{\sum_i X_i}(1-\theta)^{n-{\sum_i X_i}} \\ 
  &=& \theta^{n \bar{X}}(1-\theta)^{n-{n \bar{X}}} 
 \end{eqnarray*}
 \item Then we maximise to get $\bar{\theta}^{ML} = \bar{X}$ 
\end{itemize}
\end{frame}

\begin{frame}{Example 2} 
\begin{itemize} 
 \item Recall for the wind example we model using the exponential distribution $\mathcal{E}(\theta)$ with parameter $\theta$: 
  \begin{displaymath}
  p(x) = \left\{ \begin{array}{l l}\frac{1}{\theta}  \exp^{-x/\theta} & x \geq 0 \\ 0 & x < 0 \end{array} \right. 
 \end{displaymath}
 \item Thus
  \begin{eqnarray*} 
  p(\theta; X_1, \ldots, X_n) &=& \frac{1}{\theta^n}  \exp^{-\sum_i X_i/\theta} \\ 
  &=&  \exp^{-n\log(\theta) - n\bar{X}/\theta} \\ 
 \end{eqnarray*}
 \item Then we maximise to get $\bar{\theta}^{ML} = \bar{X}$ 
\end{itemize}
\end{frame}

\begin{frame}{Properties of MLE}  
\begin{itemize} 
 \item Under certain regularity conditions on MLE we can guarantee
 \begin{itemize} 
 \item The consistency of $\bar{\theta}^{MV}$ 
 \item The asymptotic convergence rate of $1/\sqrt{n}$ 
 \end{itemize}
 \item Note that MLE is in general biased 
\end{itemize}
\end{frame}

\begin{frame}{Exercise}  
\begin{itemize} 
 \item Show that the solution for the air quality example is given by $\bar{\theta}^{MV} = \bar{X}$ by maximising $p_n(\theta; X_1, \ldots, X_n)$. 
 \begin{itemize}
 \item Hint: take $\log p_n(\theta; X_1, \ldots, X_n)$ (why?). 
 \end{itemize}
 \item Prove $\bar{\theta}^{MV} = \bar{X}$ for the second example 
\end{itemize}
\end{frame}


\begin{frame}{Summary}  
\begin{itemize} 
 \item Statistical model - $(\mathcal{X}_n, \mathcal{P}_n)$, i.i.d. assumption 
 \item Estimators - bias and convergence 
 \item Maximum Likelihood Estimation 
\end{itemize} 
\end{frame}

\end{document}