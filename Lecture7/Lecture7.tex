\documentclass{beamer}

\usepackage{beamerthemesplit}
\usepackage[utf8x]{inputenc}
\usepackage{pgf}
\usepackage{default}
\usepackage{url}
\usepackage{subfigure}
\usepackage{algorithmic} 
\usepackage{algorithm} 
\usepackage{amssymb}
\usepackage{graphicx}
\usepackage{booktabs}
\usepackage{mathtools}

\usetheme{Singapore}


%Define some commands for printing correct variables in math mode 
\newcommand{\av}{\textbf{a}}
\newcommand{\bv}{\textbf{b}}
\newcommand{\cv}{\textbf{c}}
\newcommand{\dv}{\textbf{d}}
\newcommand{\ev}{\textbf{e}}
\newcommand{\fv}{\textbf{f}}
\newcommand{\gv}{\textbf{g}}
\newcommand{\hv}{\textbf{h}}
\newcommand{\iv}{\textbf{i}}
\newcommand{\jv}{\textbf{j}}
\newcommand{\kv}{\textbf{k}}
\newcommand{\lv}{\textbf{l}}
\newcommand{\mv}{\textbf{m}}
\newcommand{\nv}{\textbf{n}}
\newcommand{\ov}{\textbf{o}}
\newcommand{\pv}{\textbf{p}}
\newcommand{\qv}{\textbf{q}}
\newcommand{\rv}{\textbf{r}}
\newcommand{\sv}{\textbf{s}}
\newcommand{\tv}{\textbf{t}}
\newcommand{\uv}{\textbf{u}}
\newcommand{\vv}{\textbf{v}}
\newcommand{\wv}{\textbf{w}}
\newcommand{\xv}{\textbf{x}}
\newcommand{\yv}{\textbf{y}}
\newcommand{\zv}{\textbf{z}}

\newcommand{\alphav}{\mbox{\boldmath$\alpha$}}
\newcommand{\betav}{\mbox{\boldmath$\beta$}}
\newcommand{\gammav}{\mbox{\boldmath$\gamma$}}
\newcommand{\xiv }{\mbox{\boldmath$\xi$}}
\newcommand{\muv}{\mbox{\boldmath$\mu$}}
\newcommand{\tauv}{\mbox{\boldmath$\tau$}}
\newcommand{\Omegam}{\mbox{\boldmath$\Omega$}}
\newcommand{\Lambdam}{\mbox{\boldmath$\Lambda$}}
\newcommand{\Sigmam}{\mbox{\boldmath$\Sigma$}}
\newcommand{\Gammam}{\mbox{\boldmath$\Gamma$}}
\newcommand{\Deltam}{\mbox{\boldmath$\Delta$}}
\newcommand{\Thetam}{\mbox{\boldmath$\Theta$}}
\newcommand{\Phim}{\mbox{\boldmath$\Phi$}}
\newcommand{\Pim}{\mbox{\boldmath$\Pi$}}

\newcommand{\diag}{\mbox{diag}}
\newcommand{\tr}{\mbox{tr}}
\newcommand{\card}{\mbox{card}}
\newcommand{\cov}{\mbox{cov}}
\newcommand{\sign}{\mbox{sign}}
\newcommand{\var}{\mbox{var}}
\newcommand{\st}{\mbox{s.t.}}
\newcommand{\rank}{\mbox{rank}}
\newcommand{\argmin}{\mbox{argmin}}
\newcommand{\argmax}{\mbox{argmax}}

\newcommand{\Am}{\textbf{A}}
\newcommand{\Bm}{\textbf{B}}
\newcommand{\Cm}{\textbf{C}}
\newcommand{\Dm}{\textbf{D}}
\newcommand{\Em}{\textbf{E}}
\newcommand{\Fm}{\textbf{F}}
\newcommand{\Gm}{\textbf{G}}
\newcommand{\Hm}{\textbf{H}}
\newcommand{\Imat}{\textbf{I}}
\newcommand{\Jm}{\textbf{J}}
\newcommand{\Km}{\textbf{K}}
\newcommand{\Lm}{\textbf{L}}
\newcommand{\Mm}{\textbf{M}}
\newcommand{\Nm}{\textbf{N}}
\newcommand{\Om}{\textbf{O}}
\newcommand{\Pm}{\textbf{P}}
\newcommand{\Qm}{\textbf{Q}}
\newcommand{\Rm}{\textbf{R}}
\newcommand{\Sm}{\textbf{S}}
\newcommand{\Tm}{\textbf{T}}
\newcommand{\Um}{\textbf{U}}
\newcommand{\Vm}{\textbf{V}}
\newcommand{\Wm}{\textbf{W}}
\newcommand{\Xm}{\textbf{X}}
\newcommand{\Ym}{\textbf{Y}}
\newcommand{\Zm}{\textbf{Z}}

%Use regular expression: (\[a-z])([^a-zA-Z])  -> \1v\2  to change old style macros 
\graphicspath{{./Figures/}}

\title{Statistics and the Analysis of Data\\ Lecture 7: Parametric Estimation I}
\author{Charanpal Dhanjal \\ \texttt{charanpal@gmail.com}} 
\institute{\'{E}cole des Ponts}
\date{7th January 2014}

\begin{document}

\frame{\titlepage}

\begin{frame}{Recap}  
\begin{itemize} 
\item Difference types of convergence - almost surely, mean, probability 
\item Strong law of large numbers 
\item Central Limit Theorem 
\end{itemize}
\end{frame}

\begin{frame}{Outline}  
\begin{itemize} 
 \item Parametric estimation 
 \item Bias, convergence
 \item Maximum likelihood estimation
\end{itemize}
\end{frame}

\begin{frame}{Introduction}
\begin{itemize}
 \item Imagine we have some observations of a random variable with an unknown distribution. Assume we know a set of distributions that might generate the observations. 
 \item How can we find the correct distribution? 
 \item For example, if we know the variable is distributed using the Poisson distribution with rate $\theta$ 
 \item This is the central aim of \emph{parametric estimation}
\end{itemize}
\end{frame}

\begin{frame}{Parametric Estimation} 
\begin{itemize} 
 \item Three types of problem 
 \begin{itemize} 
 \item Estimate the unknown parameter 
 \item Determine a region in which we know the parameter lies with a certain probability 
 \item Given a set of values $\rho$ in which the parameter lies, decide from the observations if the parameter lies within $\rho$ 
 \end{itemize} 
\end{itemize}
\end{frame}

\begin{frame}{An Example} 
\begin{itemize}
 \item Want to find the distribution governing the ATMO index in Paris which rates air quality from 1 (best) to 10 (worst). 
 \item Look at the case in which this index passes 8, and sample $n$ days in order to get some measurements. 
 \item The observations are denoted $X_1, \ldots, X_n$ and each case is 0 if the AMTO index is less than or equal to 8, otherwise 1. Assuming the random variable is Bernoulli one, we can write 
\begin{displaymath}
P(X = 0) = p^* \quad P(X = 1) = 1- p^*  
\end{displaymath}
for some $p^* \in (0, 1)$. 
\end{itemize}
\end{frame}

\begin{frame}{Statistical Models} 
\begin{itemize} 
 \item We have a set of observations $X_1, \ldots, X_n$
 \item A \emph{statistical model} is a collection of distribution functions $\mathcal{P} = \{P_{n, \theta}, \theta \in \Theta\}$ where $\theta$ is a parameter and $n$ is the number of observations 
 \begin{itemize} 
  \item For example $\mathcal{P}$ could be a set of $n$ Gaussian distributions $\{\mathcal{N}(\mu_1, \sigma^2_1), \mathcal{N}(\mu_2, \sigma^2_2), \ldots, \mathcal{N}(\mu_n, \sigma^2_n)\}$ with $\mu_i \in \mathbb{R}$ and $\sigma_i > 0$ for all $i$  
 \end{itemize}
 \item In general it is impossible to infer the distribution given the observations without assuming a certain structure 
 \item Here, we assume the observations are i.i.d. 
\end{itemize}
\end{frame}

\begin{frame}{Estimators} 
\begin{itemize} 
\item We can write $X_1, \ldots, X_n \sim_{i.i.d.} P_{\theta^*}$ to mean we draw all observations from a distribution $P$ with unknown parameter $\theta^* \in \Theta$
\item We call $\theta^*$ the \emph{true value of the parameter}
\item An \emph{estimator} is a function $\rho: \mathbb{R}^n \rightarrow \Theta$  
\item We would like our estimated value $\bar{\theta}_n$ as close as possible to the real value $\theta^*$
\end{itemize}
\end{frame}

\begin{frame}{Bias and Convergence}  
\begin{itemize} 
\item We can say that our estimated parameter is \emph{unbiased} if $\mathbb{E}[\rho(S_{\theta^*,n})] = \theta^*$ for all $\theta^* \in \Theta$ where $S_{\theta^*,n}$ is a sample of size $n$ drawn from $P_{\theta^*}$ 
\item The estimator is \emph{consistent} or \emph{convergent} if 
\begin{displaymath} 
 \lim_{n \rightarrow \infty} P_{\theta^*}(|\bar{\theta}_n - \theta^*| > \epsilon) = 0), \quad \forall \epsilon > 0, \theta^* \in \Theta 
\end{displaymath}
\item In other words the estimated parameter $\bar{\theta}_n$ approaches the true value as $n$ increases 
\end{itemize}
\end{frame}

\begin{frame}{Asymptotic Distribution} 
\begin{itemize} 
 \item We can say that our estimated parameter $\bar{\theta}_n$ is asymptotically distributed with $P_\theta^\infty$ with rate $n^{-\gamma}$ with $\gamma > 0$ if 
 \begin{displaymath}
  n^\gamma(\bar{\theta}_n - \theta^*) \xrightarrow[]{\mathcal{L}} P_\theta^\infty, \quad \forall  \theta^* \in \Theta, 
 \end{displaymath}
where $\xrightarrow[]{\mathcal{L}}$ means distribution convergence. 
\item In the special case $P_\theta^\infty$ is Gaussian $\mathcal{N}(0, \sigma^2_{\theta^*})$ we say $\bar{\theta}_n$ is \emph{asymptotically Normal} with rate $n^{-\gamma}$ and variance $\sigma^2_{\theta^*}$ 
\end{itemize}
\end{frame}

\begin{frame}{Maximum Likelihood Estimation} 
\begin{itemize} 
 \item One of the most commonly used methods for parameter estimation is \emph{maximum likelihood estimation} (MLE) 
 \item First a couple of definitions 
 \begin{displaymath}
  
 \end{displaymath}

\end{itemize}
 
\end{frame}




\end{document}