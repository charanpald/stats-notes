\documentclass[a4paper,10pt]{article}
\usepackage[utf8]{inputenc}
\usepackage{url}
\usepackage{pgf}
\usepackage{listings}
\usepackage[margin=1.5in]{geometry}

%opening
\title{Statistics and the Analysis of Data\\ Lab Session 1: An Introduction to R}
\author{Charanpal Dhanjal}
%\institute{\'{E}cole des Ponts}

\definecolor{light-gray}{gray}{0.95}

\begin{document}
\lstset{language=R, showspaces=false, showstringspaces=false, backgroundcolor=\color{light-gray}}  
\date{22nd October 2013}
\maketitle

\section{About R}

R is a software system for statistical computing and graphics, and  freely available from the site \url{http://www.r-project.org}. It is an open source software package which is developed thanks to the voluntary contributions of authors around the world. The software itself is licensed under the GNU General Public License (GPL). The R system provides a number of statistical and graphical functions which are particularly suited towards the analysis of data. 

For installation mirrors for Windows Linux and Mac OS X, download via the site \url{http://www.r-project.org/index.html}. To get started with this lab session, follow these steps: 
\begin{enumerate}
 \item Log into Windows 
 \item Start R 
 \item Try the commands listed below. Do not just copy-paste them into the R console, try to understand each of the commands. If you have any questions, please ask. 
\end{enumerate}

\section{First Steps in R} 
First try
\begin{lstlisting}
pi*sqrt(10)+exp(4)
3:10
seq(3,10)
x = c(2,3,5,7,2,1)
y = c(10,15,12)
z = c(x,y)
z^2
x*x
w=rep(x,3)
w=rep(x, each=3)
?rep # Help
ls() # List of variables
rm(x)
x
ls()
\end{lstlisting}
\noindent
Getting help 
\begin{lstlisting}
?help
help(rep)
help(demo)
demo(graphics)
\end{lstlisting}
\noindent
Matrix manipulation 
\begin{lstlisting}
x = 1:12
dim(x) = c(3,4)
?dim
x
y = matrix(1:12, nrow=3, byrow=T)
t(y)
z = matrix(1:4, nrow=2, byrow=T)
z^2
z*z
z%*%z
\end{lstlisting}
\noindent
Graphics 
\begin{lstlisting}
x = runif(50, 0, 2)
y = runif(50, 0, 2)
plot(x, y, main="Title", xlab="x", ylab="y",col="darkred")
abline(h=.6,v=.6)
text(.6,.6, "A note")
colors()
\end{lstlisting} 
\noindent
Definition of simple functions 
\begin{lstlisting}
square = function(x) x^2
square(3)
square
\end{lstlisting} 
\noindent
A more complex example 
\begin{lstlisting}
hist.norm=function(n, col)
{
x = rnorm(n)
h = hist(x, plot=F)
s = sd(x)
m = mean(x)
ylim = c(0,1.2*max(max(h$density),1/(s*sqrt(2*pi))))
xlab = "Histogram and approximation by a normal distribution"
ylab = " "
main = paste("Guassian samples : n=",n)
hist(x, freq=F, ylim=ylim, xlab=xlab, ylab=ylab, 
  col=col, main=main)
curve(dnorm(x,m,s), add=T, lwd=2)
}
op=par(mfcol=c(1,3))
hist.norm(200 , col="yellow")
hist.norm(800 , col="darkgoldenrod")
hist.norm(3200, col="blue")
par(op)
\end{lstlisting} 

\section{Simulating Random Numbers}\label{sec:randDists}
Computational modelling often uses the generation of pseudo-random numbers

\begin{lstlisting}
rnorm(10) # Generate 10 realisation using N(0,1)
rnorm(10)
rnorm(10)
plot(rnorm(100))
rbinom(10, size=20, prob=.5)
rcauchy(10)
runif(10, min=0, max=1)
sample(1:40, 5)
sample(1:10, 10, replace=T)
sample(c("fail", "success"), 10, replace=T, prob=c(0.7, 0.3))
\end{lstlisting}
\noindent
Some other distributions include: \texttt{beta, binom, cauchy, chisq, exp}, \texttt{f, gamma, norm, pois, t, unif}. 

\section{Descriptive Statistics}\label{sec:desc}

Order statistics 
\begin{lstlisting} 
x = rnorm(10) # Generate random samples 
y = sort(x) # order them 
\end{lstlisting}

\noindent
Cumulative distribution functions 
\begin{lstlisting}
x = rnorm(100)
n=length(x)
plot(sort(x), 1:n/n, type="s", ylim=c(0,1), xlab="", ylab="")
?pnorm
curve(pnorm(x,0,1), add=T, col="blue")
\end{lstlisting}

\noindent
Histograms 
\begin{lstlisting}
x = rnorm(100)
hist(x, breaks=20)
hist(x, breaks=20, freq=F, col="cyan")
curve(dnorm(x), add=T,col="darkblue")
x = rnorm(50)
h = hist(x, plot=F)
h$breaks
h$counts
?hist
\end{lstlisting}
\noindent
Boxplot
\begin{lstlisting}
x = rnorm(100)
par(mfcol=c(2,2),bg="lightcyan")
boxplot(x)
boxplot(x,horizontal=T)
boxplot(x, col="red")
boxplot(x, col="orange",border="darkblue",lwd=2)
\end{lstlisting}

\noindent
Boxplots on the same graphic 
\begin{lstlisting}
x = rnorm(100)
y = (rnorm(400))^2-1
z= rnorm(50)^3
par(bg="lightcyan")
boxplot(x,y,z,col=c("blue","white","red"),
  border=c("black","darkblue"),lwd=1.5)
\end{lstlisting}

\noindent
QQPlots: 
\begin{lstlisting}
x = rnorm(100)
y = (rnorm(400))^2-1
z = rnorm(200,m=4,sd=5)
par(bg="lightcyan",mfrow=c(2,2))
qqplot(x,y,pch=21,bg="red",fg="darkblue",lwd=2)
qqplot(x,z,pch=21,bg="red",fg="darkblue",lwd=2)
qqnorm(y,pch=21,bg="orange",fg="darkblue",lwd=2)
qqline(y,pch=21,col="blue",lwd=2)
qqnorm(z,pch=21,bg="orange",fg="darkblue",lwd=2)
qqline(z,pch=21,col="blue",lwd=2)
\end{lstlisting}

\section{Reading Data from a File} 

Data is generally stored in files and the command \texttt{read.table} allows one to read from a file. To test the command download the file \texttt{AirQuality.data} from \url{http://certis.enpc.fr/~dalalyan/StatNum.html} and place it in your working directory. Then try the following commands: 

\begin{lstlisting}
data = read.table("AirQuality.data")
# read the data 
summary(data)
# summarise the data 
hist(data$Ozone, col="gold") # histogram of the variable Ozone
attach(data)
# we can leave out the name of the data 
hist(Ozone, freq=F, col="gold")
detach(data)
hist(Ozone,freq=F,col="gold")
# error! 
\end{lstlisting}

\noindent
Some datasets are provided with R

\begin{lstlisting}
data()
# show datasets 
?WWWusage # description of data WWWusage
plot(WWWusage)
\end{lstlisting}

\section{Questions for the Assignment} 

\begin{enumerate} 
\item In the definition of the function \texttt{hist.norm}, why do we use the value \texttt{1/(s*sqrt(2*pi))}?
\item Produce the descriptive statistics (mean, standard deviation, median, min, max) for the following data: 
\begin{lstlisting} 
library(MASS)
data(geyser)
attach(geyser)
help(geyser)
\end{lstlisting}
One can use the command \texttt{summary} as well as the commands listed in the previous sections (histogram, cumulative distribution function) 
\item Comment on the parallel boxplots given in Section  \ref{sec:desc}. Regarding the three series' of data: a) Which has the most dispersion? b) Which has the most number of outliers (abnormal values)? 
\item The values in Table \ref{tab:loads} represent the maximum loads (in tonnes) supported by a cable which is made in a certain factory. 
\begin{table}[ht]
\begin{center}
\begin{tabular}{c c} 
\hline
10.1 & 12.2\\
12.2 & 12.6\\
9.3 & 11.5\\
12.4 & 9.2\\
13.7 & 14.2\\
10.8 & 11.1\\
11.6 & 13.3\\
10.1 & 11.8 \\
11.2 & 7.1\\
11.3 & 10.5\\
\hline
\end{tabular}
\end{center}
\caption{Load weights (T)}\label{tab:loads}
\end{table}
\begin{itemize}
\item What is the approximate load that can be applied to approximately 3/4 of the cables? 
\item Trace the boxplot of this data. Are there outliers? In this diagram, where can one see the values requested in the previous question? 
\item According to the boxplot, does the data seem symmetric? 
\end{itemize}
\item Generate a set of random observations $U_1 , \ldots , U_n$ and $V_1 ,\ldots , V_n$ according to a uniform distribution [0, 1] (use \texttt{runif} and a sufficiently large $n$). 
\begin{enumerate}
\item Using graphical representations to help you, guess the distribution of the variables $T_i = -2 \log U_i$. 
\item Study empirically the distribution $X_i = \sqrt{-2 \log U_i} \cos(2 \pi V_i ).$ In your opinion, what is this distribution? 
\end{enumerate}
Hint: consider the distributions given in Section \ref{sec:randDists}. 
\end{enumerate}

Please submit the report in PDF format to \texttt{charanpal at gmail.com} before midnight CET on 29/10/13 using the subject heading ``Statistics and the Analysis of Data Lab 1''. 

\end{document}
