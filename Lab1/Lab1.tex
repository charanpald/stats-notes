\documentclass[a4paper,10pt]{article}
\usepackage[utf8]{inputenc}
\usepackage{url}
\usepackage{pgf}
\usepackage{listings}

%opening
\title{Statistics and the Analysis of Data\\ Lab Session 1: Working in the R Language}
\author{Charanpal Dhanjal}
%\institute{\'{E}cole des Ponts}

\begin{document}

\maketitle


\section{About R}

R is a programming language which is freely available from the site \url{http://www.r-project.org}. More specifically, R is an open source software package which is developed thanks to the voluntary contributions of authors around the world. The software is licensed under the General Public License (GPL). 

The R system provides a number of statistical and graphical functions which are particularly suited towards the analysis of data. For installation files for Windows, download via the site \url{http://cran.us.r-project.org/bin/windows/base/}. 

To get started with this lab session, implement the following steps: 
\begin{enumerate}
 \item Log into Windows 
 \item Start R 
 \item Try the commands listed below. Do not just copy-paste them into the R console, try to each of the commands. If you have any questions, please ask the teacher. 
\end{enumerate}

\section{The first steps in R} 


\begin{itemize}
 \item First try
\begin{lstlisting}
pi*sqrt(10)+exp(4)
3:10
seq(3,10)
x = c(2,3,5,7,2,1)
y = c(10,15,12)
z = c(x,y)
z^2
x*x
w=rep(x,3)
w=rep(x, each=3)
?rep # Help
ls() # List of variables
rm(x)
x
ls()
\end{lstlisting}

\item Getting help 
\begin{lstlisting}
 ?help
help(rep)
help(demo)
demo(graphics)
\end{lstlisting}


\item Matrix manipulation 
\begin{lstlisting}
x = 1:12
dim(x) = c(3,4)
?dim
x
y = matrix(1:12, nrow=3, byrow=T)
t(y)
z = matrix(1:4, nrow=2, byrow=T)
z^2
z*z
z%*%z
\end{lstlisting}

\item Graphics 
\begin{lstlisting}
x = runif(50, 0, 2)
y = runif(50, 0, 2)
plot(x, y, main="Title", xlab="x", ylab="y",col="darkred")
abline(h=.6,v=.6)
text(.6,.6, "A note")
colors()
\end{lstlisting} 

\item Definition of functions 
\begin{itemize} 
\item Simple example 
\begin{lstlisting}
square = function(x) x^2
square(3)
square
\end{lstlisting} 

\item 
\end{itemize}
\end{document}
