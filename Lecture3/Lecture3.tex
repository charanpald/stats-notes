\documentclass{beamer}

\usepackage{beamerthemesplit}
\usepackage[utf8x]{inputenc}
\usepackage{pgf}
\usepackage{default}
\usepackage{url}
\usepackage{subfigure}
\usepackage{algorithmic} 
\usepackage{algorithm} 
\usepackage{amssymb}
\usepackage{graphicx}
\usepackage{booktabs}

\usetheme{Singapore}


%Define some commands for printing correct variables in math mode 
\newcommand{\av}{\textbf{a}}
\newcommand{\bv}{\textbf{b}}
\newcommand{\cv}{\textbf{c}}
\newcommand{\dv}{\textbf{d}}
\newcommand{\ev}{\textbf{e}}
\newcommand{\fv}{\textbf{f}}
\newcommand{\gv}{\textbf{g}}
\newcommand{\hv}{\textbf{h}}
\newcommand{\iv}{\textbf{i}}
\newcommand{\jv}{\textbf{j}}
\newcommand{\kv}{\textbf{k}}
\newcommand{\lv}{\textbf{l}}
\newcommand{\mv}{\textbf{m}}
\newcommand{\nv}{\textbf{n}}
\newcommand{\ov}{\textbf{o}}
\newcommand{\pv}{\textbf{p}}
\newcommand{\qv}{\textbf{q}}
\newcommand{\rv}{\textbf{r}}
\newcommand{\sv}{\textbf{s}}
\newcommand{\tv}{\textbf{t}}
\newcommand{\uv}{\textbf{u}}
\newcommand{\vv}{\textbf{v}}
\newcommand{\wv}{\textbf{w}}
\newcommand{\xv}{\textbf{x}}
\newcommand{\yv}{\textbf{y}}
\newcommand{\zv}{\textbf{z}}

\newcommand{\alphav}{\mbox{\boldmath$\alpha$}}
\newcommand{\betav}{\mbox{\boldmath$\beta$}}
\newcommand{\gammav}{\mbox{\boldmath$\gamma$}}
\newcommand{\xiv }{\mbox{\boldmath$\xi$}}
\newcommand{\muv}{\mbox{\boldmath$\mu$}}
\newcommand{\tauv}{\mbox{\boldmath$\tau$}}
\newcommand{\Omegam}{\mbox{\boldmath$\Omega$}}
\newcommand{\Lambdam}{\mbox{\boldmath$\Lambda$}}
\newcommand{\Sigmam}{\mbox{\boldmath$\Sigma$}}
\newcommand{\Gammam}{\mbox{\boldmath$\Gamma$}}
\newcommand{\Deltam}{\mbox{\boldmath$\Delta$}}
\newcommand{\Thetam}{\mbox{\boldmath$\Theta$}}
\newcommand{\Phim}{\mbox{\boldmath$\Phi$}}
\newcommand{\Pim}{\mbox{\boldmath$\Pi$}}

\newcommand{\diag}{\mbox{diag}}
\newcommand{\tr}{\mbox{tr}}
\newcommand{\card}{\mbox{card}}
\newcommand{\cov}{\mbox{cov}}
\newcommand{\sign}{\mbox{sign}}
\newcommand{\var}{\mbox{var}}
\newcommand{\st}{\mbox{s.t.}}
\newcommand{\rank}{\mbox{rank}}
\newcommand{\argmin}{\mbox{argmin}}
\newcommand{\argmax}{\mbox{argmax}}

\newcommand{\Am}{\textbf{A}}
\newcommand{\Bm}{\textbf{B}}
\newcommand{\Cm}{\textbf{C}}
\newcommand{\Dm}{\textbf{D}}
\newcommand{\Em}{\textbf{E}}
\newcommand{\Fm}{\textbf{F}}
\newcommand{\Gm}{\textbf{G}}
\newcommand{\Hm}{\textbf{H}}
\newcommand{\Imat}{\textbf{I}}
\newcommand{\Jm}{\textbf{J}}
\newcommand{\Km}{\textbf{K}}
\newcommand{\Lm}{\textbf{L}}
\newcommand{\Mm}{\textbf{M}}
\newcommand{\Nm}{\textbf{N}}
\newcommand{\Om}{\textbf{O}}
\newcommand{\Pm}{\textbf{P}}
\newcommand{\Qm}{\textbf{Q}}
\newcommand{\Rm}{\textbf{R}}
\newcommand{\Sm}{\textbf{S}}
\newcommand{\Tm}{\textbf{T}}
\newcommand{\Um}{\textbf{U}}
\newcommand{\Vm}{\textbf{V}}
\newcommand{\Wm}{\textbf{W}}
\newcommand{\Xm}{\textbf{X}}
\newcommand{\Ym}{\textbf{Y}}
\newcommand{\Zm}{\textbf{Z}}

%Use regular expression: (\[a-z])([^a-zA-Z])  -> \1v\2  to change old style macros 
\graphicspath{{./Figures/}}

\title{Statistics and the Analysis of Data\\ Lecture 3: An Introduction to Probability Theory}
\author{Charanpal Dhanjal \\ \texttt{charanpal@gmail.com}} 
\institute{\'{E}cole des Ponts}
\date{5th November 2013}

\begin{document}

\frame{\titlepage}


\begin{frame}{Recap Part I}  
\begin{itemize} 
\item PCA is method to deal with multivariate data 
\item Represent observations in matrix form 
\begin{displaymath} 
 \textbf{X} = \left[\begin{array}{c c c} x_{11} & \ldots & x_{1d} \\ 
                     \vdots & \ddots & \vdots \\
 x_{n1} & \ldots & x_{nd}  \end{array} \right]  = \left[\begin{array}{c} \xv_1^T \\ \vdots \\  \xv_n^T \end{array} \right]
\end{displaymath}
\item PCA forms a linear combination of columns of $\textbf{X}$ that are uncorrelated (``hidden'' variables)
\end{itemize}
\end{frame}


\begin{frame}{Recap Part II}
\begin{itemize} 
 \item First we must centre and normalise data 
\item We want a $k$ dimensional projection that captures most variance such that $\Vm^T\Vm = \Imat$
 \begin{displaymath}
  \max \phi(\Vm) = \max \sum_{i=1}^n \|\xv_{i}^T\Vm\Vm^T \|^2
 \end{displaymath}
 \item To solve this find the eigenvectors (denote by columns of $\Vm$) corresponding to largest eigenvalues of covariance matrix $\Cm = \frac{1}{n}\Xm^T\Xm$ 
 \item New features are $\Zm = \Xm\Vm$, known as \emph{Principal Components}
\end{itemize}
\end{frame}

\begin{frame}{Motivations}
\begin{itemize} 
 \item Probabilities are everywhere 
 \begin{itemize}
 \item Games
 \item Passing genes from parent to child 
 \item Medical trials
 \item Behaviour of groups of people 
 \end{itemize} 
\item These events can be modelled 
\item Applications: search engines, insurance, energy demand prediction etc. 
\end{itemize}
\end{frame}

\begin{frame}{What is Probability?} 
\begin{itemize}
 \item A fair coin is governed probabilistically. Repeated tosses give unpredictable results. 
 \item Some formalities 
 \begin{itemize}
 \item An \emph{experiment} gives us an observation of the random event
 \item The \emph{sample space} $S$ is the set of all outcomes of an experiment e.g. $S = \{H, T\}$ for coin toss $S = \{1,2,3,4,5,6\}$ for a die  
 \end{itemize} 
 \item To find the probability of an event we repeat the same experiment approaching an infinite number of times 
 \item Sometimes even large numbers of experiments are difficult e.g. effect of firearms ban on gun crime, effect of austerity/stimulus on recessions 
\end{itemize}
\end{frame}

\begin{frame}{Definitions}
\begin{itemize}
 \item Let $A \subseteq S$ then the probability of $A$ occurring is denoted by $P(A) \in [0, 1]$
 \item Some properties 
 \begin{itemize}
 \item $P(\emptyset) = 0$
 \item $P(S) = 1$
 \item $P(A^c) = 1 - P(A) \quad \forall A$  where $A^c = S \backslash A$
 \item Let $A_1, A_2, \ldots, A_k$ be disjoint sets i.e. $A_i \cap A_j = \emptyset, i \neq j$, then 
 \begin{displaymath} 
  P\left(\bigcup_{i=1}^k A_i \right) = \sum_{i=1}^k P(A_i)
 \end{displaymath}

 \end{itemize} 
\end{itemize}
 
\end{frame}


\end{document}